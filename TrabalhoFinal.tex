% Options for packages loaded elsewhere
\PassOptionsToPackage{unicode}{hyperref}
\PassOptionsToPackage{hyphens}{url}
\PassOptionsToPackage{dvipsnames,svgnames,x11names}{xcolor}
%
\documentclass[
  letterpaper,
  DIV=11,
  numbers=noendperiod]{scrartcl}

\usepackage{amsmath,amssymb}
\usepackage{iftex}
\ifPDFTeX
  \usepackage[T1]{fontenc}
  \usepackage[utf8]{inputenc}
  \usepackage{textcomp} % provide euro and other symbols
\else % if luatex or xetex
  \usepackage{unicode-math}
  \defaultfontfeatures{Scale=MatchLowercase}
  \defaultfontfeatures[\rmfamily]{Ligatures=TeX,Scale=1}
\fi
\usepackage{lmodern}
\ifPDFTeX\else  
    % xetex/luatex font selection
\fi
% Use upquote if available, for straight quotes in verbatim environments
\IfFileExists{upquote.sty}{\usepackage{upquote}}{}
\IfFileExists{microtype.sty}{% use microtype if available
  \usepackage[]{microtype}
  \UseMicrotypeSet[protrusion]{basicmath} % disable protrusion for tt fonts
}{}
\makeatletter
\@ifundefined{KOMAClassName}{% if non-KOMA class
  \IfFileExists{parskip.sty}{%
    \usepackage{parskip}
  }{% else
    \setlength{\parindent}{0pt}
    \setlength{\parskip}{6pt plus 2pt minus 1pt}}
}{% if KOMA class
  \KOMAoptions{parskip=half}}
\makeatother
\usepackage{xcolor}
\setlength{\emergencystretch}{3em} % prevent overfull lines
\setcounter{secnumdepth}{-\maxdimen} % remove section numbering
% Make \paragraph and \subparagraph free-standing
\ifx\paragraph\undefined\else
  \let\oldparagraph\paragraph
  \renewcommand{\paragraph}[1]{\oldparagraph{#1}\mbox{}}
\fi
\ifx\subparagraph\undefined\else
  \let\oldsubparagraph\subparagraph
  \renewcommand{\subparagraph}[1]{\oldsubparagraph{#1}\mbox{}}
\fi

\usepackage{color}
\usepackage{fancyvrb}
\newcommand{\VerbBar}{|}
\newcommand{\VERB}{\Verb[commandchars=\\\{\}]}
\DefineVerbatimEnvironment{Highlighting}{Verbatim}{commandchars=\\\{\}}
% Add ',fontsize=\small' for more characters per line
\usepackage{framed}
\definecolor{shadecolor}{RGB}{241,243,245}
\newenvironment{Shaded}{\begin{snugshade}}{\end{snugshade}}
\newcommand{\AlertTok}[1]{\textcolor[rgb]{0.68,0.00,0.00}{#1}}
\newcommand{\AnnotationTok}[1]{\textcolor[rgb]{0.37,0.37,0.37}{#1}}
\newcommand{\AttributeTok}[1]{\textcolor[rgb]{0.40,0.45,0.13}{#1}}
\newcommand{\BaseNTok}[1]{\textcolor[rgb]{0.68,0.00,0.00}{#1}}
\newcommand{\BuiltInTok}[1]{\textcolor[rgb]{0.00,0.23,0.31}{#1}}
\newcommand{\CharTok}[1]{\textcolor[rgb]{0.13,0.47,0.30}{#1}}
\newcommand{\CommentTok}[1]{\textcolor[rgb]{0.37,0.37,0.37}{#1}}
\newcommand{\CommentVarTok}[1]{\textcolor[rgb]{0.37,0.37,0.37}{\textit{#1}}}
\newcommand{\ConstantTok}[1]{\textcolor[rgb]{0.56,0.35,0.01}{#1}}
\newcommand{\ControlFlowTok}[1]{\textcolor[rgb]{0.00,0.23,0.31}{#1}}
\newcommand{\DataTypeTok}[1]{\textcolor[rgb]{0.68,0.00,0.00}{#1}}
\newcommand{\DecValTok}[1]{\textcolor[rgb]{0.68,0.00,0.00}{#1}}
\newcommand{\DocumentationTok}[1]{\textcolor[rgb]{0.37,0.37,0.37}{\textit{#1}}}
\newcommand{\ErrorTok}[1]{\textcolor[rgb]{0.68,0.00,0.00}{#1}}
\newcommand{\ExtensionTok}[1]{\textcolor[rgb]{0.00,0.23,0.31}{#1}}
\newcommand{\FloatTok}[1]{\textcolor[rgb]{0.68,0.00,0.00}{#1}}
\newcommand{\FunctionTok}[1]{\textcolor[rgb]{0.28,0.35,0.67}{#1}}
\newcommand{\ImportTok}[1]{\textcolor[rgb]{0.00,0.46,0.62}{#1}}
\newcommand{\InformationTok}[1]{\textcolor[rgb]{0.37,0.37,0.37}{#1}}
\newcommand{\KeywordTok}[1]{\textcolor[rgb]{0.00,0.23,0.31}{#1}}
\newcommand{\NormalTok}[1]{\textcolor[rgb]{0.00,0.23,0.31}{#1}}
\newcommand{\OperatorTok}[1]{\textcolor[rgb]{0.37,0.37,0.37}{#1}}
\newcommand{\OtherTok}[1]{\textcolor[rgb]{0.00,0.23,0.31}{#1}}
\newcommand{\PreprocessorTok}[1]{\textcolor[rgb]{0.68,0.00,0.00}{#1}}
\newcommand{\RegionMarkerTok}[1]{\textcolor[rgb]{0.00,0.23,0.31}{#1}}
\newcommand{\SpecialCharTok}[1]{\textcolor[rgb]{0.37,0.37,0.37}{#1}}
\newcommand{\SpecialStringTok}[1]{\textcolor[rgb]{0.13,0.47,0.30}{#1}}
\newcommand{\StringTok}[1]{\textcolor[rgb]{0.13,0.47,0.30}{#1}}
\newcommand{\VariableTok}[1]{\textcolor[rgb]{0.07,0.07,0.07}{#1}}
\newcommand{\VerbatimStringTok}[1]{\textcolor[rgb]{0.13,0.47,0.30}{#1}}
\newcommand{\WarningTok}[1]{\textcolor[rgb]{0.37,0.37,0.37}{\textit{#1}}}

\providecommand{\tightlist}{%
  \setlength{\itemsep}{0pt}\setlength{\parskip}{0pt}}\usepackage{longtable,booktabs,array}
\usepackage{calc} % for calculating minipage widths
% Correct order of tables after \paragraph or \subparagraph
\usepackage{etoolbox}
\makeatletter
\patchcmd\longtable{\par}{\if@noskipsec\mbox{}\fi\par}{}{}
\makeatother
% Allow footnotes in longtable head/foot
\IfFileExists{footnotehyper.sty}{\usepackage{footnotehyper}}{\usepackage{footnote}}
\makesavenoteenv{longtable}
\usepackage{graphicx}
\makeatletter
\def\maxwidth{\ifdim\Gin@nat@width>\linewidth\linewidth\else\Gin@nat@width\fi}
\def\maxheight{\ifdim\Gin@nat@height>\textheight\textheight\else\Gin@nat@height\fi}
\makeatother
% Scale images if necessary, so that they will not overflow the page
% margins by default, and it is still possible to overwrite the defaults
% using explicit options in \includegraphics[width, height, ...]{}
\setkeys{Gin}{width=\maxwidth,height=\maxheight,keepaspectratio}
% Set default figure placement to htbp
\makeatletter
\def\fps@figure{htbp}
\makeatother
\newlength{\cslhangindent}
\setlength{\cslhangindent}{1.5em}
\newlength{\csllabelwidth}
\setlength{\csllabelwidth}{3em}
\newlength{\cslentryspacingunit} % times entry-spacing
\setlength{\cslentryspacingunit}{\parskip}
\newenvironment{CSLReferences}[2] % #1 hanging-ident, #2 entry spacing
 {% don't indent paragraphs
  \setlength{\parindent}{0pt}
  % turn on hanging indent if param 1 is 1
  \ifodd #1
  \let\oldpar\par
  \def\par{\hangindent=\cslhangindent\oldpar}
  \fi
  % set entry spacing
  \setlength{\parskip}{#2\cslentryspacingunit}
 }%
 {}
\usepackage{calc}
\newcommand{\CSLBlock}[1]{#1\hfill\break}
\newcommand{\CSLLeftMargin}[1]{\parbox[t]{\csllabelwidth}{#1}}
\newcommand{\CSLRightInline}[1]{\parbox[t]{\linewidth - \csllabelwidth}{#1}\break}
\newcommand{\CSLIndent}[1]{\hspace{\cslhangindent}#1}

\KOMAoption{captions}{tableheading}
\makeatletter
\makeatother
\makeatletter
\makeatother
\makeatletter
\@ifpackageloaded{caption}{}{\usepackage{caption}}
\AtBeginDocument{%
\ifdefined\contentsname
  \renewcommand*\contentsname{Table of contents}
\else
  \newcommand\contentsname{Table of contents}
\fi
\ifdefined\listfigurename
  \renewcommand*\listfigurename{List of Figures}
\else
  \newcommand\listfigurename{List of Figures}
\fi
\ifdefined\listtablename
  \renewcommand*\listtablename{List of Tables}
\else
  \newcommand\listtablename{List of Tables}
\fi
\ifdefined\figurename
  \renewcommand*\figurename{Figure}
\else
  \newcommand\figurename{Figure}
\fi
\ifdefined\tablename
  \renewcommand*\tablename{Table}
\else
  \newcommand\tablename{Table}
\fi
}
\@ifpackageloaded{float}{}{\usepackage{float}}
\floatstyle{ruled}
\@ifundefined{c@chapter}{\newfloat{codelisting}{h}{lop}}{\newfloat{codelisting}{h}{lop}[chapter]}
\floatname{codelisting}{Listing}
\newcommand*\listoflistings{\listof{codelisting}{List of Listings}}
\makeatother
\makeatletter
\@ifpackageloaded{caption}{}{\usepackage{caption}}
\@ifpackageloaded{subcaption}{}{\usepackage{subcaption}}
\makeatother
\makeatletter
\@ifpackageloaded{tcolorbox}{}{\usepackage[skins,breakable]{tcolorbox}}
\makeatother
\makeatletter
\@ifundefined{shadecolor}{\definecolor{shadecolor}{rgb}{.97, .97, .97}}
\makeatother
\makeatletter
\makeatother
\makeatletter
\makeatother
\ifLuaTeX
  \usepackage{selnolig}  % disable illegal ligatures
\fi
\IfFileExists{bookmark.sty}{\usepackage{bookmark}}{\usepackage{hyperref}}
\IfFileExists{xurl.sty}{\usepackage{xurl}}{} % add URL line breaks if available
\urlstyle{same} % disable monospaced font for URLs
\hypersetup{
  pdftitle={Simulação probabilística da TIR de um parque eólico},
  pdfauthor={Sérgio Duarte},
  colorlinks=true,
  linkcolor={blue},
  filecolor={Maroon},
  citecolor={Blue},
  urlcolor={Blue},
  pdfcreator={LaTeX via pandoc}}

\title{Simulação probabilística da TIR de um parque eólico}
\author{Sérgio Duarte}
\date{2024-02-12}

\begin{document}
\maketitle
\ifdefined\Shaded\renewenvironment{Shaded}{\begin{tcolorbox}[interior hidden, borderline west={3pt}{0pt}{shadecolor}, boxrule=0pt, enhanced, sharp corners, frame hidden, breakable]}{\end{tcolorbox}}\fi

\hypertarget{resumo}{%
\section{Resumo}\label{resumo}}

\hypertarget{enquadramento}{%
\section{Enquadramento}\label{enquadramento}}

Uma correcta avaliação da viabilidade económica de um projecto de
energia eólica é seguramente um dos pontos mais importantes no
desenvolvimento do mesmo Castro (\protect\hyperlink{ref-intro}{2012}).
Esta avaliação representa um desafio uma vez que o cálculo da mesma está
necessariamente assente em processos estocásticos: a velocidade do vento
e o preço da energia. A potência eólica disponível para extração varia
com o cubo da velocidade do vento, pelo que aproximar o cálculo
energético através de valores médios pode levar erros grosseiros que
mascaram a incerteza associada ao projecto{[}\textbf{ref:
Introdução}{]}.

Modelar a incerteza deste processo pode então revelar-se crucial para
uma melhor avaliação do risco associado ao projecto.

\hypertarget{contexto-financeiro}{%
\subsection{Contexto financeiro}\label{contexto-financeiro}}

Um projecto de produção de energia deverá ser avaliado financeiramente
como qualquer investimento. Desta forma, torna-se importante
contabilizar todos os encargos e recebimentos associados ao projecto, de
modo a calcular o mapa de fluxo de caixa (\emph{cashflow}).

Importa clarificar algumas definições antes de avançar para a
metodologia de cálculo.

Taxa de atualização: \(a = [(1+T1)(1+T2)(1+T3)]-1\),

onde:

\(T1:\) taxa de rendimento real

\(T2:\) taxa de risco

\(T3:\) taxa de inflação

No caso presente a taxa de inflação será assumida como nula, pois afeta
recebimentos e pagamentos (chamada de avaliação a preços constantes).

Os indicadores comuns de avaliação de investimentos são:

\begin{itemize}
\tightlist
\item
  Valor Atual Líquido (VAL);
\item
  Taxa Interna de Rendibilidade (TIR);
\end{itemize}

O VAL é dado por:

\[
VAL = \sum_{j=i}^n\frac{R_{Lj}}{(1+a)^j} - \sum_{j=0}^{n-1} \frac{I_j}{(1+a)^j}
\]

onde n é o tempo de vida útil do investimento em anos, \(R_{Lj}\) é o
retorno anual líquido, dado por \(R_j-Operaçao\&Manutenção\) e \(I_j\)
representa a anuidade do investimento no ano j. Este valor representa o
retorno esperado no final do tempo de vida útil do investimento. A TIR
representa a taxa de atualização que anula o VAL, ou seja, a taxa de
rendibilidade mínima necessária para cobrir o investimento efetuado.
Este valor pode ser obtido através da optimização da seguinte expressão:
\[
\sum_{j=i}^n\frac{R_{Lj}}{(1+TIR)^j} - \sum_{j=0}^{n-1} \frac{I_j}{(1+TIR)^j} = 0
\]

Para calcular estes indicadores é necessário obter dados de:

\begin{itemize}
\tightlist
\item
  Investimento inicial
\item
  O\&M
\item
  Retorno
\end{itemize}

No presente estudo será assumido que não existem regimes bonificados e
toda a energia é vendida a preço de mercado, pelo que este preço deverá
ser obtido de uma distribuição anual de preços.

\hypertarget{contexto-energuxe9tico}{%
\subsection{Contexto energético}\label{contexto-energuxe9tico}}

O retorno de um investimento eólico está directamente relacionado com o
recurso disponível na área de instalação do projeto. Tipicamente, a
velocidade do vento segue uma distribuição de Weibull, tal que
\(X \sim Weibull(\beta, \theta)\), onde \(\beta\) é o factor de forma e
\(\theta\) o factor de escala da distribuição. Esta distribuição tem a
seguinte função densidade de probabilidade: \[
f(x|{\beta, \theta}) = \frac{\beta}{\theta^\beta}x^{\beta-1}e^{-(\frac{x}{\theta})^\beta}
\]

\includegraphics{TrabalhoFinal_files/figure-pdf/unnamed-chunk-3-1.pdf}

Com base nesta distribuição podemos simular as velocidade de vento
esperadas durante o tempo de vida do projecto e assim estimar a energia
produzida. Através da curva de potência característica de uma turbina
eólica é possível converter a velocidade de vento numa potência.

\includegraphics{TrabalhoFinal_files/figure-pdf/unnamed-chunk-4-1.pdf}

Uma vez obtidos valores para a potência elétrica produzida pela turbina,
a energia anual produzida pode ser calculada por:

\[
E_a = \int^{u_{max}}_{u_0}f(u) P(u)du
\] Os elementos expostos permitem então proceder a uma simulção
probabilística da TIR, obtendo um intervalo de confiança para a mesma,
de forma a modela a incerteza inerente ao projecto.

\hypertarget{metodologia}{%
\section{Metodologia}\label{metodologia}}

\hypertarget{arquitetura-da-simulauxe7uxe3o}{%
\subsection{Arquitetura da
simulação}\label{arquitetura-da-simulauxe7uxe3o}}

\textbf{Objectivo: Obter intervalo de confiança para TIR, dados os
custos iniciais e vida útil do projecto}

Procedimentos preliminares:

\begin{itemize}
\item
  Obter custos iniciais

  \begin{itemize}
  \item
    Custos de instalação
  \item
    Custos dos equipamentos
  \item
    Custos de O\&M
  \item
    Taxa de atualização -\textgreater{} como definir?
  \item
    definir como vai ser pago o investimento
  \end{itemize}
\item
  Obter dados velocidade de vento

  \begin{itemize}
  \tightlist
  \item
    Ajustar distribuição de Weibull
  \end{itemize}
\item
  Obter dados de mercado

  \begin{itemize}
  \tightlist
  \item
    Ajustar distribuição de Weibull
  \end{itemize}
\item
  Obter curva de potência da turbina selecionada
\end{itemize}

Simulação de Monte Carlo (granularidade
horária,~\emph{N}~horas,~\emph{i} iterações):

\begin{itemize}
\item
  Correr~\emph{i} vezes:

  \begin{itemize}
  \item
    Simular velocidades de vento
  \item
    Aplicar curva de potência ao vector de velocidade de vento para
    obter a potência horária

    \begin{itemize}
    \tightlist
    \item
      Como os segmentos são horários, a potência equivale à energia E =
      P * dt, dt = 1
    \end{itemize}
  \item
    Simular preço de energia
  \item
    Multiplicação elementar dos vectores energia e vento
  \item
    Fazer agregação anual
  \item
    Fazer balanço
  \item
    Calcular taxa interna de rendibilidade
  \item
    Guardar resultado
  \end{itemize}
\item
  Obter distribuição de TIR
\end{itemize}

\hypertarget{recolha-de-dados}{%
\subsection{Recolha de Dados}\label{recolha-de-dados}}

Antes iniciar a recolha de dados foi escolhido um local fictício para a
implantação do parque eólico a estudar, de modo a obter os dados de
velocidades de vento correspondentes.

\textbf{{[}INSERIR IMAGEM DA GRELHA{]}}

\hypertarget{velocidade-de-vento}{%
\subsubsection{Velocidade de vento}\label{velocidade-de-vento}}

Os dados de velocidades de vento são provenientes do produto ERA 5
referente a reanálises atmosférias, fornecido pelo ECMWF Hersbach, H.,
et. al (\protect\hyperlink{ref-era5}{2017}). Estes dados consistem numa
série temporal das componentes Norte e Oeste das velocidades de vento,
numa grelha horizontal, a uma altitude de 100m, em formato GRIB.

\textbf{{[}INSERIR IMAGEM DE EXEMPLO{]}}

De modo a utilizaros dados, as componentes do vento foram combinadas
para obter a norma da velocidade do vento:

\[
\overline V = |U,V| = \sqrt{U^2 + V^2}
\]

Foi feita a média espacial da grelha, para cada instante da série, e os
dados foram convertidos para formato CSV, para facilitar a sua
utilização.

\hypertarget{preuxe7o-de-energia}{%
\subsubsection{Preço de energia}\label{preuxe7o-de-energia}}

Os dados de preços de energia foram obtidos através do \emph{website} da
OMIE (operador do mercado Ibérico de energia), referentes ao ano de
2022{[}\textbf{REF}{]}. Os preços são apresentados numa série temporal,
em formato CSV, em €/MWh.

\hypertarget{curva-de-potuxeancia}{%
\subsubsection{Curva de potência}\label{curva-de-potuxeancia}}

LIVRO

\hypertarget{custos-de-investimento}{%
\subsubsection{Custos de investimento}\label{custos-de-investimento}}

LIVRO

\hypertarget{preparauxe7uxe3o-dos-dados-para-a-simulauxe7uxe3o}{%
\subsection{Preparação dos dados para a
simulação}\label{preparauxe7uxe3o-dos-dados-para-a-simulauxe7uxe3o}}

\hypertarget{distribuiuxe7uxe3o-de-vento}{%
\subsubsection{Distribuição de
vento}\label{distribuiuxe7uxe3o-de-vento}}

Os parâmetros da função densidade de probabilidade da velocidade de
vento foram estimados a partir dos dados recolhidos, utilizando o
estimador de máxima verosimilhança. O estimador foi minimzado
numericamente recorrendo à função \textbf{\texttt{optim}} do R.

\hypertarget{estimador-de-muxe1xima-verosimilhanuxe7a}{%
\paragraph{Estimador de máxima
verosimilhança}\label{estimador-de-muxe1xima-verosimilhanuxe7a}}

O estimador é dado por:

\begin{aligned}
f(x|{\beta, \theta}) &= \frac{\beta}{\theta^\beta}x^{\beta-1}e^{-(\frac{x}{\theta})^\beta} \Rightarrow 
L(\beta, \theta | X) 
&= f(x_1|\beta, \theta)f(x_2|\beta, \theta)...f(x_n|\beta, \theta) 
&= \prod^n_{i=0} \frac{\beta}{\theta}\left(\frac{x_i}{\theta}\right)^{(\beta-1)} e^{-\left(\frac{x_i}{\theta}\right)^\beta} 
&= \left(\frac{\beta}{\theta}\right)^n \prod^n_{i=0}x_i^{(\beta-1)}e^{-\left(\frac{x_i}{\theta}\right)^\beta}
M(\beta, \theta|X) &= \log{\left(L(\beta, theta|X)\right)} 
&= \log{\left(\left(\frac{\beta}{\theta}\right)^n\prod^n_{i=0}x_i^{(\beta-1)}e^{-\left(\frac{x_i}{\theta}\right)^\beta}\right)} 
&= n\left[\log{\beta}-\log{\theta}\right] + \sum^n_{i=0} \left( (\beta-1)\log{x_i} - \left(\frac{x_i}{\theta}\right)^\beta\right)
\end{aligned}

De onde obtemos a seguinte distribuição:

\includegraphics{TrabalhoFinal_files/figure-pdf/unnamed-chunk-5-1.pdf}

Visualmente verificamos um bom ajuste aos dados. O teste de
\emph{Kolmogorov-Smirnov} corrobora isto, não havendo evidência para
rejeitar a hipótese nula de que as duas distribuições são diferentes.

\begin{verbatim}

    Asymptotic two-sample Kolmogorov-Smirnov test

data:  rvs.vento and velocidade.vento$velocidade_de_vento
D = 0.0097412, p-value = 0.1651
alternative hypothesis: two-sided
\end{verbatim}

\hypertarget{distribuiuxe7uxe3o-de-preuxe7os}{%
\subsubsection{Distribuição de
preços}\label{distribuiuxe7uxe3o-de-preuxe7os}}

Os dados de preços da energia eléctrica utilizados foram referentes ao
ano de 2022. Estes não seguem uma distribuição comum, pelo que a mesmo
foi aproximada com uma mistura de 3 distruições normais, através da
bilbioteca \textbf{\texttt{mixtools}} do R.

A escolha do número 3 para o número de distribuições na mistura deveu-se
sobretudo aos resultados empíricos dos vários testes realizados. No
entanto este valor não é descabido pois podemos racionalizá-lo da
seguinte forma: existem períodos em que a produção de energia é elevada,
o que faz baixar o preço; períodos em que é reduzida, o que faz aumentar
o preço; e períodos em que tende para a principal moda da distribuição,
que faz com o preço se aproxime da média, gerando assim 3 processos de
geração de preços.

\includegraphics{TrabalhoFinal_files/figure-pdf/unnamed-chunk-8-1.pdf}

Apesar de a função de densidade de probabilidade gerada não ser
perfeira, verificamos que é muito próxima e pelo teste de \emph{KS}
verificamos que não existe evidência suficiente para rejeitar a hipótese
nula de igualdade de distribuições.

\begin{verbatim}

    Asymptotic two-sample Kolmogorov-Smirnov test

data:  precos.energia$preco and rvs.preco
D = 0.018493, p-value = 0.09997
alternative hypothesis: two-sided
\end{verbatim}

\hypertarget{cuxe1lculo-da-tir}{%
\subsubsection{Cálculo da TIR}\label{cuxe1lculo-da-tir}}

descrever cálculo

Custo inicial de instalação e O\&M

\begin{quote}
Em termos gerais, pode afirmar-se que, para Portugal, o investimento
unitário total poderá variar entre um valor médio-baixo de 1000 €/kW e
um valor médio-alto de 1500 €/kW, sendo o investimento unitário médio
reportado de 1297 €/kW ({[}Fonte: International Energy Agency{]})*. Para
os encargos de O\&M, um valor médio entre 1\% e 2\% do investimento
total é correntemente usado.
\end{quote}

selecionar turbina

\begin{itemize}
\item
  altura
\item
  curva de potência
\item
  custo
\end{itemize}

\hypertarget{resultados-e-discussuxe3o}{%
\section{\texorpdfstring{\textbf{Resultados e
discussão}}{Resultados e discussão}}\label{resultados-e-discussuxe3o}}

\hypertarget{simulauxe7uxe3o}{%
\subsubsection{Simulação}\label{simulauxe7uxe3o}}

\begin{Shaded}
\begin{Highlighting}[]
\NormalTok{n.sim }\OtherTok{\textless{}{-}} \DecValTok{100} \CommentTok{\# }
\NormalTok{potencia.turbina }\OtherTok{\textless{}{-}} \DecValTok{1} \CommentTok{\# MW}
\NormalTok{numero.turbinas }\OtherTok{\textless{}{-}} \DecValTok{10}
\NormalTok{tempo.vida }\OtherTok{\textless{}{-}} \DecValTok{20} \CommentTok{\# anos}
\NormalTok{taxa.de.atualização }\OtherTok{\textless{}{-}}\NormalTok{ .}\DecValTok{05}

\NormalTok{retorno }\OtherTok{\textless{}{-}} \FunctionTok{replicate}\NormalTok{(n.sim, }
                      \FunctionTok{simulacao.retorno}\NormalTok{(potencia.turbina, }
\NormalTok{                                        numero.turbinas, }
\NormalTok{                                        tempo.vida, }
\NormalTok{                                        taxa.de.atualização, }
\NormalTok{                                        params.vento, }
\NormalTok{                                        params.preco))}

\FunctionTok{hist}\NormalTok{(retorno)}
\end{Highlighting}
\end{Shaded}

\begin{figure}[H]

{\centering \includegraphics{TrabalhoFinal_files/figure-pdf/unnamed-chunk-10-1.pdf}

}

\end{figure}

\hypertarget{conclusuxe3o}{%
\section{Conclusão}\label{conclusuxe3o}}

\hypertarget{bibliografia}{%
\section{Bibliografia}\label{bibliografia}}

\url{https://cds.climate.copernicus.eu/cdsapp\#!/dataset/reanalysis-era5-single-levels?tab=form}

\url{https://www.r-bloggers.com/2018/08/structuring-r-projects/}

\url{https://github.com/ecmwf/eccodes-python?tab=readme-ov-file}

\url{https://docs.xarray.dev/en/stable/user-guide/time-series.html}

\url{https://github.com/ecmwf/cfgrib}

\url{https://www.r-bloggers.com/2019/08/maximum-likelihood-estimation-from-scratch/}

\url{https://www.uaar.edu.pk/fs/books/12.pdf}

\url{https://scholarsarchive.byu.edu/cgi/viewcontent.cgi?article=3508\&context=etd}

\hypertarget{refs}{}
\begin{CSLReferences}{1}{0}
\leavevmode\vadjust pre{\hypertarget{ref-intro}{}}%
Castro, Rui. 2012. \emph{Uma Introdução Às Energias Renováveis: Eólica,
Fotovoltaica e Mini-Hídrica}. 2nd ed. IST Press.

\leavevmode\vadjust pre{\hypertarget{ref-era5}{}}%
Hersbach, H., et. al. 2017. {``Complete ERA5 from 1940: Fifth Generation
of ECMWF Atmospheric Reanalyses of the Global Climate. Copernicus
Climate Change Service (C3S) Data Store (CDS). DOI:
10.24381/Cds.143582cf.''}

\end{CSLReferences}



\end{document}
